\documentclass[a4paper, 10pt]{report}
\usepackage[italian]{babel}
\usepackage[T1]{fontenc}
\usepackage[utf8]{inputenc}
\usepackage{charter}
\usepackage{amsmath}
\usepackage{amsthm}
\usepackage{amsfonts}
\usepackage{graphicx}
\usepackage{wrapfig}
\usepackage{tcolorbox}
\usepackage{fancyhdr}
\usepackage{longtable}
\usepackage{tikz}
\usetikzlibrary{patterns}
\usetikzlibrary{shapes,positioning,calc}
\usetikzlibrary{shapes.geometric, arrows}
\usetikzlibrary{arrows.meta}

\usepackage{geometry}
\geometry{a4paper, left=2cm,right=2cm,top=2cm,bottom=2cm}

\pagestyle{fancy}
\chead{}
\rhead{\bfseries 11 novembre 2019}
\lhead{\bfseries Basi di dati}

\begin{document}

\subsection*{Valori nulli}
Non sempre nella base di dati esistono tutti i valori per ogni specifica tupla. Infatti può capitare che:
\begin{itemize}
\item[-] Il valore di quell'attributo sia inesistente per quella tupla;
\item[-] Il valore di quell'attributo sia sconosciuto (in quel momento) per quella tupla.
\end{itemize}

\noindent Per poter gestire queste situazioni viene iterodotto il valore nullo -> ora gli attributi di una tupla possono assumere un valore entro il loro dominio o il valore nullo.

\paragraph*{Definizione} Una tupla su X è una funzione:
			\[
				t : \: X \rightarrow \{NULL\} \cup ( \bigcup_{A\in X} DOM (A))
			\]
			dove:
			\[
				t [ A ] = v \in DOM (A) \vee t[A] = NULL
			\]

\noindent La presenza di valori nulli è comunque accettata solo per specifici attributi; non è possibile avere tuple di soli valori nulli; negli attributi replicati la presenza di valori nulli può rendere inutilizzabile l'informazione.

\subsection*{Vincoli di integrità}
Poichè può essere necessario introdurre vincoli sulla popolazione (istanza) di una base di dati, ad esempio perchè non tutte le possibili istanze sono corrette
rispetto al sistema informativo considerato, è stato introdotto il concetto di vincolo di integrità.

\noindent Un vincolo di integrità è una condizione (espressa da un predicato)
che deve essere sempre soddisfatta
da ogni istanza della base di dati.\\

\noindent Esempio\\

\begin{tikzpicture}[relation/.style={rectangle split, rectangle split parts=#1, rectangle split part align=base, draw, anchor=center, align=center, text height=3mm, text centered}]\hspace*{-0.3cm}
			
			% RELATIONS
			
			\node (countrytitle) {\textbf{TRENO}};
			
			\node [relation=7, rounded corners, rectangle split horizontal, rectangle split part fill={lightgray!50}, anchor=north west, below=0.6cm of countrytitle.west, anchor=west] (proprietà)
			{%
				\nodepart{one} Numero
				\nodepart{two} OraPart
				\nodepart{three} MinutoPart
				\nodepart{four} Categoria
				\nodepart{five} Destinazione
				\nodepart{six} OraArr
				\nodepart{seven} MinutoArr};
			
			\end{tikzpicture}
			
			\begin{tikzpicture}[relation/.style={rectangle split, rectangle split parts=#1, rectangle split part align=base, draw, anchor=center, align=center, text height=3mm, text centered}]\hspace*{-0.3cm}
			
			% RELATIONS
			
			\node (countrytitle) {\textbf{FERMATA}};
			
			\node [relation=4, rounded corners, rectangle split horizontal, rectangle split part fill={lightgray!50}, anchor=north west, below=0.6cm of countrytitle.west, anchor=west] (proprietà)
			{%
				\nodepart{one} NumTreno
				\nodepart{two} Stazione
				\nodepart{three} OraFer
				\nodepart{four} MinutoFer};
			
			\end{tikzpicture}

\noindent \\Vincoli:
\begin{itemize}
				\item[-] $\forall t \in TRENO : \: t[\textup{OraPart}] \in \{0,1,\dots, 23\}$;
				\item[-] $\forall t \in TRENO : \: t[\textup{MinutoPart}] \in \{0,1,\dots, 59\}$;
				\item[-] $\forall t \in TRENO : \: t[\textup{Numero}] > 0$;
				\item[-] $\forall t \in TRENO : \: t[\textup{Numero}] > 5000 \Rightarrow t[\textup{Categoria}] = \textup{'regionale'}$;
				\item[-] $\forall t, t' \in TRENO : \: t \neq t' \Rightarrow 
				t[\textup{Numero}] \neq t' [\textup{Numero}] $;
				\item[-] $\forall f \in FERMATA : \: \exists t \in TRENO 
				: \: t[\textup{Numero}] = f [\textup{NumTreno}] $;
				\item[-] $\forall t \in TRENO : \: t[\textup{Categoria}] = \textup{'regionale'} \Rightarrow 
				\exists f \in FERMATA : \: f [\textup{NumTreno}] = t[\textup{Numero}] $.
			\end{itemize}

\noindent I vincoli possono essere classificati come segue:

\begin{longtable}{| p{.15\textwidth} | p{.75\textwidth} |}
\textbf{Vincoli di dominio} & Impongono una restrizione sul dominio dell'attributo
			di una relazione.
			
			\medskip
			
			Esempio
			
			\medskip
			\begin{center}
			\begin{tikzpicture}[relation/.style={rectangle split, rectangle split parts=#1, rectangle split part align=base, draw, anchor=center, align=center, text height=3mm, text centered}]\hspace*{-0.3cm}
			
			% RELATIONS
			
			\node (countrytitle) {\textbf{TRENO}};
			
			\node [relation=7, rounded corners, rectangle split horizontal, rectangle split part fill={lightgray!50}, anchor=north west, below=0.6cm of countrytitle.west, anchor=west] (proprietà)
			{%
				\nodepart{one} Numero
				\nodepart{two} OraPart
				\nodepart{three} MinutoPart
				\nodepart{four} Categoria
				\nodepart{five} Destinazione
				\nodepart{six} OraArr
				\nodepart{seven} MinutoArr};
			
			\end{tikzpicture}
			\end{center}
\end{longtable}

\begin{longtable}{| p{.15\textwidth} | p{.75\textwidth} |}		
 & I vincoli per il dominio di un attributo
			di TRENO sono:
			\begin{itemize}
				\item[-] $\forall t \in TRENO : \:
				 t[\textup{OraPart}] \in \{0,1,\dots, 23\}$;
				\item[-] $\forall t \in TRENO : \:
				t[\textup{MinutoPart}] \in \{0,1,\dots, 59\}$;
				\item[-] $\forall t \in TRENO : \:
				t[\textup{Numero}] > 0$.
			\end{itemize}
\\\\
\textbf{Vincoli di tupla} & Impongono una restrizione alla combinazione di valori
			che una tupla della relazione può assumere
			indipendentemente dalle altre tuple.
			
			\medskip
			
			Esempio
			
			\medskip
		
			\begin{center}
			\begin{tikzpicture}[relation/.style={rectangle split, rectangle split parts=#1, rectangle split part align=base, draw, anchor=center, align=center, text height=3mm, text centered}]\hspace*{-0.3cm}
			
			% RELATIONS
			
			\node (countrytitle) {\textbf{TRENO}};
			
			\node [relation=7, rounded corners, rectangle split horizontal, rectangle split part fill={lightgray!50}, anchor=north west, below=0.6cm of countrytitle.west, anchor=west] (proprietà)
			{%
				\nodepart{one} Numero
				\nodepart{two} OraPart
				\nodepart{three} MinutoPart
				\nodepart{four} Categoria
				\nodepart{five} Destinazione
				\nodepart{six} OraArr
				\nodepart{seven} MinutoArr};
			
			\end{tikzpicture}
			\end{center}
			
			Il vincolo di tupla è:
			\[
				\forall t \in TRENO : \:
				t[\textup{Numero}] > 5000 \Rightarrow t[\textup{Categoria}] = 
				\textup{'regionale'}
			\]		
\\\\
\textbf{Vincoli intrarelazionali} & Impongono una restrizione al contenuto di una
			relazione e specificano una condizione che ogni tupla
			della relazione deve soddisfare rispetto alle altre tuple
			della medesima relazione.
			
			Una sottocategoria importante di tali vincoli include i
			\textbf{vincoli di chiave}:
			\begin{itemize}
				\item[-] Superchiave;
				\item[-] Chiave candidata;
				\item[-] Chiave primaria.
			\end{itemize}

\textbf{Superchiave}: data una relazione di schema $R(X)$, un insieme di attributi $K$, sottoinsieme di $X$, è SUPERCHIAVE per
				$R(X)$ se, per ogni istanza $r$ di $R(X)$ vale la seguente
				condizione:
				\[
					\forall t, t' \in r : 
					\: t \neq t' \Rightarrow t[K] \neq t' [K]
				\]
				dove:
				\[
					t[K] \neq t' [K] \equiv \exists A_i \in K : \:
					 t[A_i] \neq t' [A_i]
				\]
				
\textbf{Chiave candidata}: data una relazione di schema $R(X)$, un insieme di
				attributi $K$, sottoinsieme di $X$, è CHIAVE CANDIDATA
				(o CHIAVE) per $R(X)$, se $K$ è superchiave per $R(X)$ e
				vale la seguente condizione:
				\[
					\neg \exists K' \subset K : \: K' 
					\textup{è superchiave per} R(X)
				\]
			
				Esiste sempre una chiave candidata $K$ per una
				relazione $R(X)$.	
	
\medskip	
				
\textbf{Chiave primaria}: data una relazione di schema $R(X)$ la sua CHIAVE
				PRIMARIA è la chiave candidata scelta per \textbf{identificare
				le tuple della relazione}.
				
				Una chiave primaria $K$ ha le seguenti caratteristiche:
				\begin{itemize}
					\item Non contiene mai valori nulli
					\item Su $K$ il sistema genera una struttura d'accesso ai dati
					(o indice) per supportare le interrogazioni.
				\end{itemize}
				
				La chiave primaria si rappresenta sottolineando l'attributo.							
\end{longtable}
\newpage
\begin{longtable}{| p{.15\textwidth} | p{.78\textwidth} |}
 & Esempio 1 
\medskip
\begin{center}
\begin{tikzpicture}[relation/.style={rectangle split, rectangle split parts=#1, rectangle split part align=base, draw, anchor=center, align=center, text height=3mm, text centered}]\hspace*{-0.3cm}
			
			% RELATIONS
			
			\node (countrytitle) {\textbf{TRENO}};
			
			\node [relation=7, rounded corners, rectangle split horizontal, rectangle split part fill={lightgray!50}, anchor=north west, below=0.6cm of countrytitle.west, anchor=west] (proprietà)
			{%
				\nodepart{one} Numero
				\nodepart{two} OraPart
				\nodepart{three} MinutoPart
				\nodepart{four} Categoria
				\nodepart{five} Destinazione
				\nodepart{six} OraArr
				\nodepart{seven} MinutoArr};
			
			\end{tikzpicture}
			\medskip
			\begin{tikzpicture}[relation/.style={rectangle split, rectangle split parts=#1, rectangle split part align=base, draw, anchor=center, align=center, text height=3mm, text centered}]\hspace*{-0.3cm}
			
			% RELATIONS
			
			\node (countrytitle) {\textbf{FERMATA}};
			
			\node [relation=4, rounded corners, rectangle split horizontal, rectangle split part fill={lightgray!50}, anchor=north west, below=0.6cm of countrytitle.west, anchor=west] (proprietà)
			{%
				\underline{NumTreno}
				\nodepart{two}\underline{Stazione}
				\nodepart{three} OraFer
				\nodepart{four} MinutoFer};
			\end{tikzpicture}
			
\end{center}
			
\noindent Considerando i seguenti sottoinsiemi di $X$, dire se, rispetto al
			contesto applicativo, il sottoinsieme è superchiave o no e, se è
			superchiave, dire se è anche chiave:
			\begin{itemize}
				\item $K_1=\{\textup{Numero}\}$
				\item $K_2=\{\textup{Numero, Categoria}\}$
				\item $K_3=\{\textup{OraPart, MinutoPart, Destinazione, Categoria}\}$
			\end{itemize}	
			
Esempio 2
\medskip

\begin{center}
\begin{tikzpicture}[relation/.style={rectangle split, rectangle split parts=#1, rectangle split part align=base, draw, anchor=center, align=center, text height=3mm, text centered}]\hspace*{-0.3cm}
			
			% RELATIONS
			
			\node (countrytitle) {\textbf{TRENO}};
			
			\node [relation=7, rounded corners, rectangle split horizontal, rectangle split part fill={lightgray!50}, anchor=north west, below=0.6cm of countrytitle.west, anchor=west] (proprietà)
			{%
				\underline{Numero}
				\nodepart{two} OraPart
				\nodepart{three} MinutoPart
				\nodepart{four} Categoria
				\nodepart{five} Destinazione
				\nodepart{six} OraArr
				\nodepart{seven} MinutoArr};
			
			\end{tikzpicture}
\end{center}
			\\\\
\textbf{Vincoli interrelazionali} & Impongono una restrizione al contenuto di una
			relazione e specificano una condizione che ogni tupla
			della relazione deve soddisfare rispetto alle tuple di
			altre relazioni della base di dati.
			
			Una sottocategoria importante di tali vincoli include i
			vincoli di integrità referenziale (o vincoli sulle
			chiavi esportate).
			
			\medskip
			
\textbf{Vincoli di integrità referenziale}: Un vincolo di integrità referenziale tra un insieme di
				attributi $Y=\{A_1 , \dots, A_p \}$ di $R_1$ e un insieme di attributi $K=\{K_1 , \dots, K_p \}$, chiave primaria di un'altra relazione $R_2$, è soddisfatto se, per ogni istanza $r_1$ di $R_1$ e per ogni
				istanza $r_2$ di $R_2$ vale la seguente condizione:
				\[
					\forall t \in r_1 : \: \exists s \in r_2 : \:
					 \forall i \in \{1,\dots, p\} : \: t[A_i] = s[K _i]
				\]	
				
				\medskip
				
\textbf{Vincoli di integrità referenziale con valori nulli}: Un vincolo di integrità referenziale tra un insieme di
				attributi $Y=\{A_1 , \dots, A_p \}$ di $R_1$ e un insieme di attributi $K=\{K_1 , \dots, K_p \}$, chiave primaria di un'altra relazione $R_2$, è soddisfatto se, per ogni istanza $r_1$ di $R_1$ e per ogni istanza $r_2$ di $R_2$ vale la seguente condizione:
				\[
					\forall t \in r_1 : \: \exists s \in r_2 : \:
					 (\forall i \in  \{1,\dots, p\} : \: t[A_i ] = s[K_i ]) \vee
				(\exists i \in \{1,\dots, p\} : \: t[A_i ] = NULL)
				\]
				
Il vincolo di integrità referenziale si rappresenta tramite una freccia che parte dall'attributo e punto alla tabello di origine dell'attributo stesso.

\medskip

Esempio

\medskip

\begin{flushright}
\begin{tikzpicture}[relation/.style={rectangle split, rectangle split parts=#1, rectangle split part align=base, draw, anchor=center, align=center, text height=3mm, text centered}]\hspace*{-0.3cm}
			
			% RELATIONS
			
			\node (countrytitle) {\textbf{TRENO}};
			
			\node [relation=7, rounded corners, rectangle split horizontal, rectangle split part fill={lightgray!50}, anchor=north west, below=0.6cm of countrytitle.west, anchor=west] (treno)
			{%
				\underline{Numero}
				\nodepart{two} OraPart
				\nodepart{three} MinutoPart
				\nodepart{four} Categoria
				\nodepart{five} Destinazione
				\nodepart{six} OraArr
				\nodepart{seven} MinutoArr};
			
			\node  [below=1.3cm of treno.west, anchor=west] (fermatatitle) {\textbf{FERMATA}};
			
			\node [relation=4, rounded corners, rectangle split horizontal, rectangle split part fill={lightgray!50}, anchor=north west, below=0.6cm of fermatatitle.west, anchor=west] (fermata)
			{%
				\underline{NumTreno}
				\nodepart{two}\underline{Stazione}
				\nodepart{three} OraFer
				\nodepart{four} MinutoFer};
			
			% FOREIGN KEYS
			
			\draw[-latex] (fermata.one south) -- ++(0,-0.2) -| ($(fermata.one south) + (12,0)$) |- ($(countrytitle.east) + (0,0)$);
			\end{tikzpicture}
\end{flushright}

				
									 
\end{longtable}









\end{document}