\documentclass[a4paper, 10pt]{report}
\usepackage[italian]{babel}
\usepackage[T1]{fontenc}
\usepackage[utf8]{inputenc}
\usepackage{charter}
\usepackage{amsmath}
\usepackage{amsthm}
\usepackage{amsfonts}
\usepackage{graphicx}
\usepackage{wrapfig}
\usepackage{tcolorbox}
\usepackage{fancyhdr}
\usepackage{longtable}
\usepackage{tikz}
\usetikzlibrary{patterns}
\usetikzlibrary{shapes,positioning,calc}
\usetikzlibrary{shapes.geometric, arrows}
\usetikzlibrary{arrows.meta}

\usepackage{geometry}
\geometry{a4paper, left=2cm,right=2cm,top=2cm,bottom=2cm}

\pagestyle{fancy}
\chead{}
\rhead{\bfseries 11 novembre 2019}
\lhead{\bfseries Basi di dati}

\begin{document}

\section*{\underline{Progettazione logica}}
La progettazione logica consiste nella traduzione dello schema ER in uno specifico modello logico, in questo caso quello relazionale.

\subsection*{Costrutti principali}
\begin{longtable}{| p{.20\textwidth} | p{.70\textwidth} |}
\textbf{Domini di base} & Consistono nei domini da cui si scelgono i valori delle
proprietà delle istanze di informazione da
rappresentare (es: caratteri, stringhe, interi, decimali a virgola mobile, ...).
\\\\
\textbf{Costrutto relazione} & Il costrutto relazione può essere rappresentato come una tabella formata da una lista di colonne, dove:
\begin{itemize}
\item[-] I dati sono scritti nelle righe;
\item[-] Ogni riga descrive le
caratteristiche di una istanza dell’informazione da rappresentare;
\item[-] I valori contenuti nelle colonne descrivono sempre la stessa
proprietà delle istanze di informazione da rappresentare.
\end{itemize}

Esempio:
\begin{center}
\begin{tabular}{lll}
			MILANO  & $20100$ & $1300000$ \\
			VERONA  & $37100$ & $350000$ \\
			BRESCIA & $25100$ & $250000$\\
		\end{tabular}
\end{center}

\medskip

\textbf{Definizione formale come insieme di ennuple (lists)}: dati $n$ insiemi di valori (domini) $D_1, \dots , D_n$ con $n > 0$ e indicato con $D_1 \times\dots \times D_n$ il loro prodotto cartesiano:
			
			\[
				D_1 \times \dots \times D_n = \{ (v_1, \dots, v_n) \: | \: 
				v_1 \in D_1 \wedge \dots \wedge v_n \in D_n \}
			\]
			
			una relazione $\rho$ di grado $n$ è un qualsiasi sottoinsieme di 
			$D_1 \times \dots \times D_n$ :
			\[
				\rho \subseteq D_1 \times \dots \times D_n
			\]
			dove:
			\begin{itemize}
				\item[-] $(v_1, \dots, v_n)$ è una ennupla della relazione;
				\item[-] $|\rho|$ è la cardinalità della relazione (numero di ennuple).
			\end{itemize}

I domini $D_1, \dots , D_n$ possono essere a cardinalità infinita, mentre le relazioni sono sempre a cardinalità finita; non esiste alcun ordinamento delle tuple; non è possibile duplicare una ennupla; i valori all'interno delle ennuple sono ordinati.

\medskip

Esempio

\medskip

Relazione delle città:
\begin{center}
$\rho = \{(MILANO, 20100, 1.300.000),	$
	
$(VERONA, 37100, 350.000),$

$(BRESCIA, 25100, 250.000) \}$	
	
$\rho \subseteq D_1 \times D_2 \times D_3$
	
		\begin{itemize}
			\item[-] $D_1 =$ Stringhe di caratteri;
			\item[-] $D_2 =$ Numeri interi;
			\item[-] $D_3 =$ Numeri interi.
		\end{itemize}
\end{center}
\end{longtable}

\begin{longtable}{| p{.20\textwidth} | p{.70\textwidth} |}
 & Se t è una ennupla $(v_1 , \dots, v_n )$ il valore posto in i-esima posizione si indica con la notazione:
\begin{center}
$t[i]$
\end{center}

Poichè questa modalità di accesso ai valori non è efficace per l'uso
			pratico delle relazioni, si preferisce quindi assegnare un nome
			alle colonne, portando all'introduzione della definizione di
			relazione come insieme di tuple.

\medskip
\medskip

\textbf{Definizione formale come insueme di tuple (mappings)}: sia $X$ un insieme di nomi e sia $\Delta$ l'insieme di tutti i domini di
			base ammessi dal modello. Allora si può definire una funzione
			\[
				DOM: \: X \, \rightarrow \, \Delta
			\]
			che associa ad ogni nome $A$ di $X$ un dominio $DOM(A)$ di $\Delta$. 
			I nomi di $X$ si dicono attributi.
			
			Una tupla su $X$ è una funzione:
		
			\[
				t : \: X \, \rightarrow \, \bigcup_{A \in X} DOM (A) 
			\]
			dove
			\[
				t [ A ] = v \in DOM (A)
			\]
			Una relazione su $X$ è un insieme di tuple su $X$, dove $X$ è
			l'insieme di attributi della relazione.

\medskip

Esempio

\medskip
		Relazione delle città:
		\begin{center}
		\begin{itemize}
			\item[-] X = \{Nome, CAP, Abitanti\};
			\item[-] DOM(Nome) = Stringhe di caratteri;
			\item[-] DOM(CAP) = Numeri interi;
			\item[-] DOM(Abitanti) = Numeri interi.
		\end{itemize}
		\end{center}
		\[
			\rho_X = \{ t_1 , t_2 , t_3 \}
		\]
		\begin{tabular}{lll}
			$t_1$ [Nome] $=$ MILANO & $t_2$ [Nome] $=$ VERONA & $t_3 [\dots] = \dots$ \\
			$t_1$ [CAP] $= 20100$ & $t_2$ [CAP] $= 37100$ & \\
			$t_1$ [Abitanti] $= 1.300.000$ & $t_2$ [Abitanti] $= 350.000$ & 
		\end{tabular}

\medskip

Una relazione è un insieme di tuple e quindi non può contenere tuple duplicate; i domini per gli attributi possono essere solo domini di base (
			non sono ammessi altri domini, né il prodotto cartesiano di
			domini); in generale una base di dati relazionale è costituita da più
			relazioni.

\end{longtable}

\subsection*{Progettazione}
Nel caso in cui in una relazione, ovvero una tabella, si uniscono concetti disomogenei e con un'esistenza autonoma (come raggruppare tutti i dati nella stessa tabella), si rischia di creare ridondanza, ovvero una ripetizione inutile di più tuple.

\noindent La ridondanza può, a sua volta, creare le seguenti anomelie:
\begin{itemize}
			\item Anomalia di \textbf{aggiornamento} -> per aggiornare il valore di
			un attributo si è obbligati a modificare tale valore su più tuple
			della base di dati.
			\item Anomalia di \textbf{inserimento} -> per inserire una nuova istanza
			di un concetto è necessario inserire valori al momento
			sconosciuti (sostituibili con valori nulli) per gli attributi non
			disponibili.
			\item Anomalia di \textbf{cancellazione} -> per cancellare un'istanza di un
			concetto è necessario cancellare valori ancora validi oppure
			inserire valori nulli per gli attributi da cancellare.
		\end{itemize}

\noindent Il metodo principale per eliminare le ridondanze consiste nel dividere la tabella in tabelle più piccole.\\

\noindent Esempio:\\

\begin{tikzpicture}[relation/.style={rectangle split, rectangle split parts=#1, rectangle split part align=base, draw, anchor=center, align=center, text height=3mm, text centered}]\hspace*{-0.3cm}
			
			% RELATIONS
			
			\node (countrytitle) {\textbf{PROPRIETÀ}};
			
			\node [relation=9, rounded corners, rectangle split horizontal, rectangle split part fill={lightgray!50}, anchor=north west, below=0.6cm of countrytitle.west, anchor=west] (proprietà)
			{%
				\nodepart{one}   CodiceFiscale
				\nodepart{two} Cognome
				\nodepart{three} Nome
				\nodepart{four} DataNas
				\nodepart{five} CodiceCatasto
				\nodepart{six} Via
			    \nodepart{seven} NumeroCivico 
			    \nodepart{eight} Subalterno
			    \nodepart{nine} Tipo	};
		
		\end{tikzpicture}

\noindent \\La tabella unica PROPRIETÀ può essere divisa come segue:\\

\begin{tikzpicture}[relation/.style={rectangle split, rectangle split parts=#1, rectangle split part align=base, draw, anchor=center, align=center, text height=3mm, text centered}]\hspace*{-0.3cm}
			
			% RELATIONS
			
			\node (countrytitle) {\textbf{PROPRIETARIO}};
			
			\node [relation=4, rounded corners, rectangle split horizontal, rectangle split part fill={lightgray!50}, anchor=north west, below=0.6cm of countrytitle.west, anchor=west] (proprietà)
			{%
				\nodepart{one} CodiceFiscale
				\nodepart{two} Cognome
				\nodepart{three}Nome
				\nodepart{four}DataNas};

		 \end{tikzpicture}


\begin{tikzpicture}[relation/.style={rectangle split, rectangle split parts=#1, rectangle split part align=base, draw, anchor=center, align=center, text height=3mm, text centered}]\hspace*{-0.3cm}
		
			% RELATIONS
			
			\node (countrytitle) {\textbf{UNITÀ\_IMMOBILIARE}};
			
			\node [relation=5, rounded corners, rectangle split horizontal, rectangle split part fill={lightgray!50}, anchor=north west, below=0.6cm of countrytitle.west, anchor=west] (proprietà)
			{%
				\nodepart{one}CodiceCatasto
				\nodepart{two}Via
				\nodepart{three}NumeroCivico 
				\nodepart{four}Subalterno
				\nodepart{five}Tipo	};
			
		\end{tikzpicture}

\noindent \\La suddivisione dell'esempio sopra, però, non riesce a rappresentare la stessa informazione della tabella unica, sebbene ne elimini la ridondanza. Infatti non conserva l'informazione che descrive
			l'associazione tra proprietario e unità immobiliare.
			
\noindent Per mantenere anche questa informazione è necessario aggiungere una terza tabella che, per sua natura, conterrà delle ridondanze (seppur limitate).\\

\noindent Esempio\\

\begin{tikzpicture}[relation/.style={rectangle split, rectangle split parts=#1, rectangle split part align=base, draw, anchor=center, align=center, text height=3mm, text centered}]\hspace*{-0.3cm}
		
		% RELATIONS
		
		\node (countrytitle) {\textbf{PROPRIETARIO}};
		
		\node [relation=4, rounded corners, rectangle split horizontal, rectangle split part fill={lightgray!50}, anchor=north west, below=0.6cm of countrytitle.west, anchor=west] (proprietà)
		{%
			\nodepart{one}   CodiceFiscale
			\nodepart{two} Cognome
			\nodepart{three} Nome
			\nodepart{four} DataNas};
		
		\end{tikzpicture}
		
		\begin{tikzpicture}[relation/.style={rectangle split, rectangle split parts=#1, rectangle split part align=base, draw, anchor=center, align=center, text height=3mm, text centered}]\hspace*{-0.3cm}
		
		% RELATIONS
		
		\node (countrytitle) {\textbf{UNITÀ\_IMMOBILIARE}};
		
		\node [relation=5, rounded corners, rectangle split horizontal, rectangle split part fill={lightgray!50}, anchor=north west, below=0.6cm of countrytitle.west, anchor=west] (proprietà)
		{%
			\nodepart{one} CodiceCatasto
			\nodepart{two} Via
			\nodepart{three} NumeroCivico
			\nodepart{four} Subalterno
			\nodepart{five} Tipo};
		
		\end{tikzpicture}
		
		\begin{tikzpicture}[relation/.style={rectangle split, rectangle split parts=#1, rectangle split part align=base, draw, anchor=center, align=center, text height=3mm, text centered}]\hspace*{-0.3cm}
		
		% RELATIONS
		
		\node (countrytitle) {\textbf{PROPRIETÀ}};
		
		\node [relation=2, rounded corners, rectangle split horizontal, rectangle split part fill={lightgray!50}, anchor=north west, below=0.6cm of countrytitle.west, anchor=west] (proprietà)
		{%
			\nodepart{one} CodiceFiscale
			\nodepart{two} CodiceCatasto};
		
		\end{tikzpicture}

\noindent \\La nuova tabella replicherà
			parte degli attributi, scegliendo tra questi quelli che
			hanno la proprietà di identificare il concetto verso il
			quale si vuole generare il legame.
			
			
\paragraph*{Modello valued - based} In base a questo, si può definire il modello relazionale come valued - based, ovvero:
\begin{itemize}
			\item[-] È totalmente indipendente dalla rappresentazione fisica (tutta
			l'informazione è nei valori) e non ci sono meccanismi per
			gestire riferimenti o puntatori tra istanze di informazione;
			\item[-] I legami logici tra tuple diverse si realizzano attraverso la
			replicazione di alcuni attributi (che hanno la proprietà di
			identificare il concetto e quindi di rappresentarlo): il legame
			tra due tuple si intende stabilito quando esse presentano gli
			stessi valori negli attributi replicati;
			\item[-] È facile trasferire i dati da una base di dati all'altra;
			\item[-]  È rappresentato solo ciò che è rilevante per l'applicazione.
		\end{itemize}
			
\subsection*{Terminologia}			
		\paragraph*{Schema di una relazione} Uno schema di una relazione è costituito dal nome della
		relazione e da un insieme di nomi per i suoi attributi:
		\[
			R(A_1 , \dots, A_n ) \quad \textup{oppure} \quad R(A_1 :\:D_1 , \dots, A_n :\: D_n )
		\]
		dove $DOM(A_i ) = D_i$
		
		\paragraph*{Schema di una base di dati} Uno schema di una base di dati è un insieme di schemi di relazione:
		\[
			S = \{R_1 (A_{1,1},\dots, A_{1,n_1} ), \dots , R_m (A_{m,1} , \dots, A_{m,n_m} )\}
		\]
		dove $R_1 \neq\dots \neq R_m$
		
		\paragraph*{Istanza di una relazione} Un'istanza di una relazione di schema $R(A_1 , \dots, A_n )$ con
		$X = \{A_1 , \dots, A_n \}$:
		è un insieme $r$ di tuple su $X$.

		\paragraph*{Istanza di una base di dati} Un'istanza di una base di dati di schema 
		$S = \{R_1 (A_{1,1} , \dots,A_{1,n_1} ), \dots, R_m (A_{m,1} , \dots, A_{m,n_m} )\}$:
		
		è un insieme di istanze di relazioni $db = \{r_1 , \dots, r_m \}$
		dove ogni $r_i$ è un'istanza della relazione di schema $R_i (A_{i,1} , \dots, A_{i,n_i} )$.
			
			
			
\end{document}