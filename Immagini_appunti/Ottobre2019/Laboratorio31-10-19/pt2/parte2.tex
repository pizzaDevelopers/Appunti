\documentclass[a4paper, 10pt]{report}
\usepackage[italian]{babel}
\usepackage[T1]{fontenc}
\usepackage[utf8]{inputenc}
\usepackage{charter}
\usepackage{amsmath}
\usepackage{amsthm}
\usepackage{amsfonts}
\usepackage{graphicx}
\usepackage{wrapfig}
\usepackage{tcolorbox}
\usepackage{fancyhdr}
\usepackage{listings}
\usepackage{longtable}
\usepackage{multicol}
\usepackage{xcolor}

\usepackage{geometry}
\geometry{a4paper, left=2cm,right=2cm,top=2cm,bottom=2cm}

\pagestyle{fancy}
\lhead{}
\chead{}
\rhead{\bfseries 31 ottobre 2019 }
\lhead{\bfseries Segnali e immagini - laboratorio}

\newcounter{main}
\setcounter{main}{1}

\lstnewenvironment{code}[1][firstnumber=\themain,name=main]
  {\lstset{language=matlab,
           basicstyle=\medium\ttfamily,
           numbers=left,
           basicstyle=\small,
           columns=fullflexible,
           inputencoding=latin1,
           #1
          }
}
{\setcounter{main}{\value{lstnumber}}}


\begin{document}

\noindent \textbf{Esempio di ricerca difetti su tessuto:}

\begin{code}
%% Cross-correlazione 2D normalizzata per trovare difetti su tessuti

%Creo le sottosezioni per i controlli
A = rgb2gray(imread('tex.jpg'));
[R,C]=size(A);
pattern1 = A(1:14,1:14);
pattern2 = A(2:15,2:15);
pattern3 = A(R-13:R,C-13:C);
pattern4 = A(R-14:R-1,C-14:C-1);
pattern5 = A(1:14,C-13:C);
pattern6 = A(2:15,C-13:C);

%bordo le sottosezioni sull'immagine principale (rappresentazione esclusivamente visiva)
figure;
imagesc(A); axis image; colormap gray; hold on;
rectangle('position',[1,1,14,14],'EdgeColor',[1 0 0]);%edgecolor colora i bordi del rettangolo con il colore indicato
rectangle('position',[2,2,15,15],'EdgeColor',[1 0 0]);
rectangle('position',[R-13,C-13,14,14],'EdgeColor',[1 0 0]);
rectangle('position',[R-14,C-14,14,14],'EdgeColor',[1 0 0]);
rectangle('position',[1,C-13,14,14],'EdgeColor',[1 0 0]);
rectangle('position',[2,C-13,14,14],'EdgeColor',[1 0 0]);

%ottengo una media della cross correlazione tra l'immagine e tutti i pattern
c1 = normxcorr2(pattern1,A);
c2 = normxcorr2(pattern2,A);
c3 = normxcorr2(pattern3,A);
c4 = normxcorr2(pattern4,A);
c5 = normxcorr2(pattern5,A);
c6 = normxcorr2(pattern6,A);

c = (c1+c2+c3+c4+c5+c6)/6;
c = c(12:end-12,12:end-12);
figure, surf(abs(c)), shading flat
figure, imagesc(abs(c)), colorbar
c=abs(c);

mask = c<0.2; %setto la soglia per il controllo se c'è difetto o no
figure, imagesc(mask)
se = strel('disk',3);
mask2 = imopen(mask,se);
figure, imagesc(mask2);

%operazioni di filtraggio morfologico (pulisco la maschera)
A=A(5:end-6,5:end-6);
A1 = A;
A1(mask2)=255;
Af=cat(3,A1,A,A);
 figure;
imshowpair(A,Af,'montage')
\end{code}

\noindent \\Analisi codice:
\begin{longtable}{| p{.25\textwidth} | p{.70\textwidth} |}

\textbf{M = size(X,DIM)} & Ritorna la lunghezza della dimensione di X specificata dallo scalare DIM.

Nel caso di un video, le quattro dimensioni sono (righe, colonne, canali colore, n. frame).
\\\\
\textbf{B = imresize(A, SCALE)} & Ritorna l'immagine A riscalata in base a SCALA.
Se SCALE = 1 ritorna l'immagine originale; se SCALE = 0.5 ritorna limmagine ridotto della metà (== prende un pixel ogni due).
\\\\
\textbf{I2 = imcrop(I)} & Assegna ad I2 una sezione dell'immagine I. La sezione viene selezionata tramite mouse.
\\\\
\textbf{J = rgb2gray(I)} & Converte l'immagine da RGB a scala di grigi.
\\\\
\textbf{[I,J] = ind2sub(SIZE,INDEX)} & Converte INDEX in una coppia di indici (I, J) basandosi su una matrice di dimensione SIZE. 

\end{longtable}

\noindent Osservazioni:
\begin{itemize}
\item[-] pattern$^*$ contiene una porzione dell'immagine che sarà usata come base per la ricerca di difetti sull'immagine stessa;
\item[-] Eseguo la cross-correlazione tra l'immagine e poi ne calcolo la media. Fatto ciò setto un li
\end{center}
\end{itemize}

\noindent Risultato:


\end{document}