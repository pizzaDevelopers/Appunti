\documentclass[a4paper, 10pt]{report}
\usepackage[italian]{babel}
\usepackage[T1]{fontenc}
\usepackage[utf8]{inputenc}
\usepackage{charter}
\usepackage{amsmath}
\usepackage{amsthm}
\usepackage{amsfonts}
\usepackage{graphicx}
\usepackage{wrapfig}
\usepackage{tcolorbox}
\usepackage{fancyhdr}
\usepackage{listings}

\usepackage{geometry}
\geometry{a4paper, left=2cm,right=2cm,top=2cm,bottom=2cm}

\pagestyle{fancy}
\lhead{}
\chead{}
\rhead{\bfseries 09 ottobre 2019 }
\lhead{\bfseries Segnali e immagini}

\begin{document}

\section*{\underline{Introduzione}}
\begin{tabular}{lp{0.60\textwidth}}
\textbf{Segnale} & una qualsiasi grandezza (fisica o astratta) che varia in un dominio (spesso il tempo, ma anche lo spazio, altro) in maniera deterministica o aleatoria, che trasporta più o meno informazione.

Esempi di segnali: segnale acustico, segnale elettronico, segnale elettromagnetico, segnale sismico, segnale su grandezze astratte (serie finanziarie, ...), immagine (segnale su due dimensioni con dirata infinita). \\\\

\textbf{Informazione} & qualcosa che quando viene fornita dissipa un’incertezza.\\\\

\textbf{Rumore} & tutto ciò che è associato al segnale, ma non porta informazione. Disturba la ricezionedel segnale e l’estrazione dell’informazione.\\\\

\textbf{Energia di un segnale} & attributo associato ad un segnale (segnale di energia), espresso in joule (potenza x tempo) -> per trasmettere un segnale serve energia.\\\\

\textbf{Potenza} & energia per unità di tempo, utile per trattare i segnali ad energia infinita.\\\\

\textbf{Signal-to-Noise Ratio (SNR)} & rapporto tra la potenza del segnale e la potenza del rumore. Più grande (di 1) è, meglio si interagisce con il segnale.\\\\

\textbf{Sistema di elaborazione dei segnali} & può essere visto come un modello matematico che, ad un segnale d’ingresso, reagisce producendo un segnale di uscita. Può essere mono o multi variabile.

Un sistema agisce su un segnale mogliorando il rapporto segnale - rumeore o estraendo informazioni dal segnale.\\\\

\textbf{Analisi in frequenza} & un segnale può essere visto, oltre che attraverso il suo sviluppo spaziale/temporale, attraverso il suo sviluppo in frequenza. Questo tipo di analisi permette di "contare" quante volte un certo evento si ripete per unità di tempo.

Più la frequenza è alta, più l'evento si ripete.\\\\

\textbf{Filtraggio} & consiste nella trasformazione di un segnale da una forma ad un'altra.\\\\

\textbf{Acquisizione di un segnale} & nel caso in cui il segnale sia un'entità esistente fisicamente, è fondamentale  essere in grado di acquisirlo. Più alta la frequenza, più costoso è l’hardware  e più difficili sono le condizioni per l'acquisizione. 

I principali strumenti impiegnati sono:
\begin{itemize}
\item[-] Sensore/trasduttore: dispositivo che acquisisce in ingresso una grandezza fisica ed esprime in uscita una grandezza elettrica il cui valore è funzione della grandezza di ingresso;
\item[-] Campionatore: trasforma un segnale da continuo a discreto;
\item[-] Digitalizzatore: discretizza/quantizza i valori assunti da un segnale.
\end{itemize}

\\\\ \textbf{Campionamento} & i segnali acquisiti dall’ambiente naturale ed alcuni trasmessi dall’uomo sono analogici (continui nello spazio/tempo e nei valori che essi trasportano -> il grafico del segnale è continuo). Il campionamento trasforma un segnale da continuo a discreto.

Se il campionamento è corretto preserva totalmente l'informazione.\\\\

\end{tabular}

\begin{tabular}{lp{0.75\textwidth}}

\textbf{Quantizzazione} & associa i valori continui di un segnale (continuo o campionato) ad un intervallo di quantizzazione, scelto tra diversi intervalli.
 Contrariamente alla campionatura (che permette di mantenere invariate le informazioni associate ad un segnale), fa perdere sempre un po’ di informazione (non più recuperabile), a causa dell’errore di quantizzazione (che si può verificare a tutte le frequenze).\\\\

\textbf{Digitalizzazione} & se un segnale viene campionato e quantizzato, allora diventa un segnale digitale.

La digitalizzazione di un segnale è essenziale per la trasmissione dello stesso.\\\\

\textbf{Filtraggio} & Il filtraggio rappresenta una famiglia di operazioni per eseguire:
\begin{itemize}
\item[-] Enhancement (miglioramento, rinforzo): migliora alcuni aspetti per aumentare l’informazione del segnale;
\item[-] Restauro: ripristina un segnale corrotto da vari tipi di rumore allo stato originale;
\item[-] Separazione: un segnale può essere visto come composizione di più segnali, e quindi scomponibile in più parti;
\item[-] Estrazione delle caratteristiche: isola le caratteristiche di un segnale. Serve principalmente per la classificazione segnale tramite machine learning.
\end{itemize}

\\\\ \textbf{Sintesi di un segnale} & consiste nella riproduzione al computer di un segnale presente in natura.

\end{tabular}


\end{document}


