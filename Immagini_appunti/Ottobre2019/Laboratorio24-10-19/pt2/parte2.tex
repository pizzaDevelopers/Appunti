\documentclass[a4paper, 10pt]{report}
\usepackage[italian]{babel}
\usepackage[T1]{fontenc}
\usepackage[utf8]{inputenc}
\usepackage{charter}
\usepackage{amsmath}
\usepackage{amsthm}
\usepackage{amsfonts}
\usepackage{graphicx}
\usepackage{wrapfig}
\usepackage{tcolorbox}
\usepackage{fancyhdr}
\usepackage{listings}
\usepackage{longtable}
\usepackage{multicol}
\usepackage{xcolor}

\usepackage{geometry}
\geometry{a4paper, left=2cm,right=2cm,top=2cm,bottom=2cm}

\pagestyle{fancy}
\lhead{}
\chead{}
\rhead{\bfseries 24 ottobre 2019 }
\lhead{\bfseries Segnali e immagini - laboratorio}

\newcounter{main}
\setcounter{main}{1}

\lstnewenvironment{code}[1][firstnumber=\themain,name=main]
  {\lstset{language=matlab,
           basicstyle=\medium\ttfamily,
           numbers=left,
           basicstyle=\small,
           columns=fullflexible,
           #1
          }
}
{\setcounter{main}{\value{lstnumber}}}


\begin{document}

\noindent \textbf{Esempio di Cross correlazione 2D:}

\begin{code}
%% Cross-correlazione 2D
% Controllate con le slide se tornano i conti con la 
 
X1 =[1     2     5     2     5;
     3     4     5     4     2;
     5     2     3     2     2;
     2     3     2     4     2;
     3     4     2     2     3];
        
X2 =[3     3     2;
     2     3     4;
     4     5     4];
 
Corr = xcorr2(X1,X2)
 
% Per fare prove ulteriori, si decommenti il codice qui sotto; 
%X1 = round(unifrnd(1,5,5,5));
%X2 = round(unifrnd(1,5,3,3));
%Corr = xcorr2(X1,X2)

\end{code}

\noindent Risultato:
\begin{code}
Corr =

     4    13    34    41    50    33    20
    16    42    80    84    84    45    18
    34    65    99    95    91    53    27
    34    62    90    94    87    54    18
    30    68    88    81    73    49    22
    16    37    45    45    44    31    12
     6    17    25    22    18    15     9
\end{code}

\end{document}