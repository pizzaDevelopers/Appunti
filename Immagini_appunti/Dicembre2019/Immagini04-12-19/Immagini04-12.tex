\documentclass[a4paper, 10pt]{report}
\usepackage[italian]{babel}
\usepackage[T1]{fontenc}
\usepackage[utf8]{inputenc}
\usepackage{charter}
\usepackage{amsmath}
\usepackage{amsthm}
\usepackage{amsfonts}
\usepackage{graphicx}
\usepackage{wrapfig}
\usepackage{tcolorbox}
\usepackage{fancyhdr}
\usepackage{listings}
\usepackage{longtable}

\usepackage{geometry}
\geometry{a4paper, left=2cm,right=2cm,top=2cm,bottom=2cm}

\pagestyle{fancy}
\lhead{}
\chead{}
\rhead{\bfseries 04 dicembre 2019 }
\lhead{\bfseries Segnali e immagini}

\begin{document}

\subsection*{Trasformata di Fourier discreta}

Dato un segnale reale continuo $f(t)$ di dominio illimitato e non periodico, campionato nel tempo con periodo (di campionamento) $\Delta T$, la sua trasformata di Fourier sarà una funzione continua, periodica (con periodo $1/\Delta T$) e con dominio illimitato:
\begin{align*}
\mathfrak{\tilde{f}}(t) = \tilde{F}(\mu) = \frac{1}{\Delta T} \sum_{n = -\infty}^{+\infty}F\left( \mu - \frac{n}{\Delta T}\right)
\end{align*}

\noindent Questa formulazione prevede però di conoscere la trsformata di Fourier teorica ($F$) del segnale di partenza, e raramente succede. Per evitare ciò, si manipola la formula fino ad ottenere
\begin{align*}
\tilde{F}(\mu) = \sum_{n = -\infty}^{+\infty}f_n e^{-j\pi \mu n \Delta T}
\end{align*} 

\paragraph*{Campionamento dello spettro} Il segnale ottenuto dall'ultima formula deve però essere campionato per poter essere rappresentato al pc.










\end{document}