\documentclass[a4paper, 10pt]{report}
\usepackage[italian]{babel}
\usepackage[T1]{fontenc}
\usepackage[utf8]{inputenc}
\usepackage{charter}
\usepackage{amsmath}
\usepackage{amsthm}
\usepackage{amsfonts}
\usepackage{graphicx}
\usepackage{wrapfig}
\usepackage{tcolorbox}
\usepackage{fancyhdr}
\usepackage{listings}
\usepackage{longtable}
\usepackage{multicol}
\usepackage{booktabs}
\usepackage{tikz}
\usepackage{charter}
\usepackage{amsmath}
\usepackage{amsthm}
\usepackage{amsfonts}
\usepackage{graphicx}
\usepackage{wrapfig}
\usepackage{tcolorbox}
\usepackage{fancyhdr}
\usepackage{listings}
\usepackage{longtable}
\usepackage{multicol}
\usepackage{xcolor}

\usepackage{geometry}
\geometry{a4paper, left=2cm,right=2cm,top=2cm,bottom=2cm}

\newcounter{main}
\setcounter{main}{1}

\lstnewenvironment{code}[1][firstnumber=\themain,name=main]
  {\lstset{language=SQL,
           basicstyle=\medium\ttfamily,
           basicstyle=\small,
           columns=fullflexible,
           #1
          }
}
{\setcounter{main}{\value{lstnumber}}}

\begin{document}

\chapter{SQL - PostgreSQL}

\noindent Il linguaggio SQL si compone di tre parti:

\begin{itemize}
\item[-] Data Definition Language (DDL), utilizzato per definire le strutture dati e la definizione dei vincoli di integrità;
\item[-] Data Manipulation Language (DML), utilizzato per inserire, aggiornare e cancellare dati. Contiene anche il Data Query Language, utilizzato per le interrogazioni;
\item[-] Data Control Languade, utilizzato per modificare la base di dati.
\end{itemize}

\noindent Un database si compone di uno o più schemi (directory), che possono contenere una o più tabelle.

\section{Data Definition Language}

\paragraph{Operazioni principali} Le operazioni principali del DDL sono:
\begin{enumerate}
\item \underline{Creazione tabella}:
\begin{code}
	CREATE TABLE nomeTabella (
    		nomeColonna TIPO VINCOLO,
    		...    		
	);
\end{code}

\item \underline{Eliminazione tabella}:
\begin{code}
	DROP TABLE nameTabella;
\end{code}

\item \underline{Modifica tabella}:
\begin{itemize}
\item[-] Inserimento nuovo attributo:
\begin{code}
	ALTER TABLE tabellaEsistente ADD COLUMN attributo tipo;
\end{code}

\item[-] Rimozione attributo:
\begin{code}
	ALTER TABLE tabellaEsistente DROP COLUMN attributo; 
\end{code}

\item[-] Modifica nome attributo:
\begin{code}
	ALTER TABLE tabellaEsistente RENAME COLUMN vecchioNome TO nuovoNome; 
\end{code}

\item[-] Aggiunta vincolo di tabella:
\begin{code}
	ALTER TABLE tabellaEsistente ADD vincolo; 
\end{code}

\item[-] Inserimento o eliminazione del vincolo associato all'attributo:
\begin{code}
	ALTER TABLE tabellaEsistente ALTER COLUMN attributo
		SET VINCOLO (valore) "oppure" DROP vincolo;
\end{code}

\item[-] Inserimento nuovo vincolo di integrità referenziale:
\begin{code}
	ALTER TABLE tabellaEsistente ADD CONSTRAINT definizioneVincolo;
\end{code}

\item[-] Rimozione nuovo vincolo di integrità referenziale:
\begin{code}
	ALTER TABLE tabellaEsistente DROP CONSTRAINT nomeVincolo;
\end{code}

\noindent Il paramentro $nomeVincolo$ è il nome dato durante la dichiarazione. Se non viene dato, il DBMS ne assegna lui uno.
\end{itemize}
\end{enumerate}

\paragraph*{Domini elementari} I domini elementari sono:
\begin{longtable}{ p{.30\textwidth}  p{.65\textwidth}}
\textbf{CHAR} & permette di rappresentare singoli caratteri o stringhe di lunghezza fissa (CHAR(20) == stringa di 20 caratteri). Se la stringa non raggiunge la dimensione fissata, il sistema aggiunge degli spezi. I caratteri devono essere dichiarati all'interno di apici singoli (' ');
\\\\
\textbf{VARCHAR(dim)} & permette di rappresentare stringhe di dimensione variabile (VARCHAR(20) indica che la dimensione max della stringa è 20 caratteri). Le stringhe devono essere dichiarate all'interno di apici singoli (' ');
\\\\
\textbf{TEXT} & Stringa di dimensione infinita (postgreSQL);
\\\\
 \textbf{BOOL} & permette di rappresentare i valori booleani. Può assumere tre valori: TRUE, FALSE, NULL;
 \\\\
 \textbf{SMALLINT} & permette di rappresentare numeri interi compresi tra -32.768 e +32.767. Equivale al tipo short di Java;
\\\\
 \textbf{INTEGET} & permette di rappresentare numeri compresi tra -2$^{31}$ e +2$^{31}$-1. Equivale al tipo int di Java;
 \vspace{0.2cm}
\\
 \textbf{REAL} & permette di rappresentare valori in virgola mobile con una precisione di 6 cifre decimali;
\\\\
 \textbf{DOUBLE PRECISION} & Come REAL, ma la precisione è di 15 cifre decimali;
\\\\
 \textbf{NUMERIC} (o \textbf{DECIMAL}) & permette di rappresentare numeri decimali in virgola fissa. Il numero è rappresentato in modo esatto (nessun errore di approssimazione in seguito a operazioni aritmetiche). 
 
Solitamente il dominio si specifica come NUMERIC(precisione, scala), dove precisione indica il numero di cifre significative (destra e sinistra della virgola), mentre scala indica il numero di cifre dopo la virgola (NUMERIC(5, 2) mi permette di rappresentare numeri come 100.01);
\\\\
 \textbf{DATE} & permette di rappresentare date nel formato YYYY-MM-DD. Le date devono essere specificate tra apici singoli (' ');
\\\\
 \textbf{TIME[(precisione)][WITH TIME ZONE]} &  permette di rappresentare un orario all'interno della giornata. La precisione permette di indicare i decimali da usare per rappresentare frazioni del secondo. 
 
WITH TIME ZONE permette di specificare un fuso orario (nomeFusoOrario o $+-$hour:minute come differenza con l'ora di Greenwich);
\\\\
  \textbf{TIMESTAMP[(precisione)][ WITH TIME ZONE]} & DATE + TIME (TIMESTAMP);
\\\\
 \textbf{INTERVAL[fields][(p)]} & permette di rappresentare una durata temporale. Il campo fields permette di limitare l'estensione dell'intervallo, mentre il campo p permette di specificate la precisione delle frazioni di secondo (se fields contiene anche i secondi). Gli intervalli possono essere specificati in più modi:
	\begin{enumerate}
	\item Formato SQL -> es: '3 4:05:06' == 3 giorni, 4 ore, 5 minuti, 6 secondi. Il massimo rappresentabile sono i giorni;
	\item Formato ISO -> es: 'P1Y2M3DT4H5S6' == 1 anno, 2 mesi, 3 giorni, 4 ore, 5 minuti, 6 secondi. La lettera P denota l'inizio della data, la lettera T l'inizio dell'ora.
\end{enumerate}
\end{longtable}
	
\paragraph*{Domini specifici} È possibile creare un dominio specifico (partendo da uno elementare) nel seguente modo:
\begin{code}
		CREATE DOMAIN nome AS tipoBase dominioBase
			[vincolo].
\end{code}
\noindent Solitamente come vincolo si usa il CHECK.
\\

\noindent Esempio:
\begin{code}
		CREATE DOMAIN giorniSettimana AS CHAR(3) 
			CHECK(VALUE IN('LUN', 'MAR', 'MER', 'GIO', 'VEN', 'SAB', 'DOM'));
\end{code}

\paragraph*{Vincoli} In base a cosa vengono associati, i vincoli si dividono in:
\begin{itemize}
\item[-] Vincoli di attributi: solitamente si specifica dopo il nome dell'attributo e si riferisce solo a quell'attributo;
\item[-] Vincoli di tabella: solitamente specificati dopo tutti gli attributi, si riferiscono a gruppi di attributi.
\end{itemize}

\noindent I vincoli principali sono:
\begin{longtable}{ p{.30\textwidth}  p{.65\textwidth}}
\textbf{DEFAULT} & si utilizza per specificare un valore di base da assegnare al campo nel caso non venga specificato. Posso assegnare anche il valore DEFAULT.
\newline

Esempio: 
\begin{code}
	prezzo NUMERIC DEFAULT 9.99;
\end{code}
\\
\textbf{CONSTRAIT} & si utilizza per assegnare un nome ad un gruppo di vincoli (es: è più semplice rimuoverli). Raggruppa tutti i vincoli che lo seguono fino alla prima virgola.
\newline

Esempio: 
\begin{code}
	price NUMERIC CONSTRAINT pPrice NOT NULL CHECK (price > 0);
\end{code}
\\
\textbf{NOT NULL} & si utilizza per indicare che il campo non può essere nullo (Attenzione: una stringa vuota non è NULL);
\newline

Esempio:
\begin{code}
	nome VARCHAR (20) NOT NULL;
\end{code}
\\
\textbf{CHECK} & si utilizza per indicare che l'attributo deve rispettare la condizione indicata nel check. Il vincolo viene rispettato se la sua espressione booleana è vero o nulla. Se l'espressione è false, il sistema dà errore;
\newline

Esempio:
\begin{code}
	stipendi NUMERIC CHECK(stipendio >= 0.0);
	CHECK(nome <> cognome);
\end{code}
\\\\
\textbf{NULL} & si utilizza per indicare che il campo deve essere nullo;
\\
\textbf{UNIQUE} & si utilizza per imporre che l'attributo sia superchiave, ovvero utti i valori assegnati a quell'attributo devono esser diversi tra loro (oppure NULL). Se si riferisce a più attributi, ogni gruppo deve avere sempre almeno un valore diverso;
\newline

Esempio:
\begin{code}
	nome VARCHAR (20) UNIQUE;
	UNIQUE (nome, cognome);
\end{code}
\\
\textbf{PRIMARY KEY} & si utilizza per identificare una chave primaria. Si ottiene dalla composizione di UNIQUE + NOT NULL. Questo attributo può essere può essere inserito una sola volta in ogni tabella.  
\newline

Esempio:
\begin{code}
	matricola CHAR(6) PRIMARY KEY;
	PRIMARY KEY(nome, cognome);
\end{code}
\end{longtable}

\paragraph*{Valore NULL} Il valore NULL di SQL è un valore speciale:
\begin{itemize}
\item[-] NULL = NULL è falso. Il NULL è sempre diverso da se stesso;
\item[-] Per eseguire confronti col NULL si devono usare specifiche operazioni.
\end{itemize}

\paragraph*{Vicoli di integrità referenziale} Un vincolo di integrità referenziale crea un legame tra i valori di un attributo, o insiemi di attributi, della tabella corrente, detta interna (o slave), e i valori di un attributo, o insieme di attributi, di un'altra tabella, detta esterna (o master). Il legame impone che:
\begin{itemize}
\item[-] Il valore dell'attributo di ogni riga della tabella interna sia presente tra quelli della tabella esterna. Se il valore nella tabella interna è NULL (es: non ho specificato NOT NULL e le ho assegnato NULL) allora il vincolo è soddisfatto;
\item[-] Il valore dell'attributo di ogni riga della tabella interna sia presente anche tra quelli della tabella esterna, a meno che questo non sia NULL. L'attributo, nella tabella esterna, deve essere vincolato da UNIQUE o PRIMARY KEY (basta che identifichi le righe della tabella esterna, non deve essere necessariamente la sua chiave primaria).
\end{itemize}

\noindent Esistono due modi per esprimere questo vincolo:
\begin{enumerate}
\item REFERENCES -> si utilizza quando il vincolo è usato solo su un attributo della tabella interna. Si dichiara come segue:
\begin{code}
	CREATE TABLE tabellaInterna(
		...
		attributo tipo REFERENCES
			tabellaEsterna(attributoChiave);
	);
\end{code}
\item FOREIGN KEY -> si utilizza  quando il vincolo è usato su più attributi della tabella interna. Si dichiara come segue:
\begin{code}
	CREATE TABLE tabellaInterna(
		...
		attributo1 tipo,
		...
		
		FOREIGN KEY(attributo1, ...) REFERENCES tabellaEsterna (nomeAttributo1Esterna, ...);
	);
\end{code}
\end{enumerate}
					\newpage
\noindent Attenzione: 
\begin{itemize}
\item[-] La tabella master deve essere creata prima della tabella slave;
\item[-] Il dominio dei due attributi collegato deve essere compatibile (meglio uguale);
\item[-] Senza l'uso di comandi specifici, un valore dell'attributo collegato può essere eliminato dalla tabella master solo se non compare nell'attributo della tabella slave.
\\

\noindent I voncoli di integrità referenziali possono essere rimossi e agginti anche dopo la creazione della tabella (vedi sopra).
\end{itemize}

\section{Data Management Language}

\paragraph*{Operazioni principali} Le operazioni principali del DML sono:
\begin{enumerate}
\item \underline{Popolazione delle tabelle}:
\begin{code}
	INSERT INTO tabellaEsistente (elencoAttributi)
		VALUES (elencoValori), 
		...
		;
\end{code}
\noindent Posso aggiungere una o più righe con ogni comando VALUES.

\item \underline{Modifica dei valori di una riga}:
\begin{code}
	UPDATE tabellaEsistente
		SET attributo = espressione,
		...
		[WHERE condizione];
\end{code}
\noindent Se non specifico il WHERE, la modifica verrà applicata a tutte le righe. Devo ripetere UPDATE per ogni modifica differente dalla precedente.

\item \underline{Eliminazione righe}:
\begin{code}
	DELETE FROM tabellaEsistente 
	[WHERE condizione];
\end{code}
\end{enumerate}

\paragraph*{Operatori SQL} Alcuni operatori particolari sono:
\begin{longtable}{ p{.30\textwidth}  p{.55\textwidth}}{Concatenzazione stringhe} & {||}
\\
{Confronto e assegnazione} & {=}
\end{longtable}

\subsection{Data Query Language}
Il processo per recuperare dati dal database è chiamato query/interrogazione.

\noindent Il comando SQL per effettuare una query è:
\begin{code}
	SELECT [ DISTINCT ]
	[ * | expression [[ AS ] output_name ] [ , ...] ]
	[ FROM from_item [ , ...] ]
	[ WHERE condition ]
	[ GROUP BY grouping_element [ , ...] ]
	[ HAVING condition [ , ...] ]
	[ { UNION | INTERSECT | EXCEPT } [ DISTINCT ]
		other_select ]
	[ ORDER BY expression [ ASC | DESC | USING operator ]]
	...
\end{code}

Nello specifico:
\begin{longtable}{ p{.20\textwidth}  p{.75\textwidth}}
\textbf{*} & ritorna tutti gli attributi delle tabelle indicate.
\\\\
\textbf{DISTINCT} & elimina le tuple risultato duplicate.
\\\\
\textbf{AS} & permette di assegnare un nome a expression (assegna un nome per la colonna risultato nella tabella risultato temporanea). I nuovi nome possono essere del tipo:
\begin{itemize}
\item[-] NOME -> viene interpretato come nome. Non è case sensitive;
\item[-] "Nome" -> viene interpretato come Nome. È case sensitive.
\end{itemize}
\\\\
\textbf{FROM} & specifica la tabella da cui prendere le tuple. Il campo from\_item può realizzare uno dei seguenti valori:
\begin{itemize}
\item[-] Prodotto cartesiano:
\begin{code}
	nomeTabella [[AS] alias [(alisaColonna [, ...])]]
\end{code}
\noindent Se sono presenti due o più tabelle, si esegue il prodotto cartesiano tra tutte le tabelle. Se ci sono attributi con lo stesso nome, nel SELECT questi vengono identificati con nomeTabella.nomeAttributo;
\item[-] Query Innstata:
\begin{code}
	(other_select) [AS] nomeRisultato [(aliasColonna [, ...])]
\end{code}
\noindent Con questa forma creo una query innestata. Il risultato della SELECT interna è la tabella con nome nomeTisultato su cui fare la query corrente;
\item[-] Join:
\begin{code}
	from_item [NATURAL] join_type from_item [ON condition]
\end{code}
\noindent Questa forma permette di selezionare un sottoinsieme del prodotto cartesiano di due o più tabelle. Il campo join\_type specifica il tipo di join. I tipi principali sono riassunti nelle slide seguenti. 
\item[-] Altre opzioni più complesse.
\end{itemize}
\\\\
\textbf{WHERE} & è seguita da una condizione booleana. La condizione può contenere:
\begin{itemize}
\item[-] Espressioni relative alle colonne richieste nella prima riga;
\item[-] \textbf{AND}, \textbf{OR}, \textbf{NOT};
\item[-] =, <>, <, >, <=, >=;
\item[-] Costanti appartenenti ai domini di base;
\item[-] \textbf{BETWEEN} val1 \textbf{AND} val2 -> ritorna true se il valore è compreso tra val1 e val2 (entrambi compresi);
\item[-] \textbf{IN}(val1, val2, ...) -> ritorna true se il valore è uguale a val2 o val2 o ...;
\item[-] \textbf{IS} [\textbf{NOT}] \textbf{NULL};
\end{itemize}  
\noindent Nella WHERE non è possibile utilizzare i nomi dell'AS assegnati nella prima riga.

\noindent Possono anche comparire i seguenti operatori:
\begin{itemize}
\item[-] LIKE -> operatore di pattern matching. Le stringhe di controllo del LIKE si compongono con due caratteri speciali:
	\begin{itemize}
	\item[-] "\_" == 1 carattere qualsiasi;
	\item[-] "\%" == 0 o più caratteri qualsiasi;
	\end{itemize}
\end{itemize}
\\
 &	\begin{itemize}
 \item[]
\begin{itemize}
 \item[]
 Esempio:
	
	\noindent Tutti gli studenti di una città che inizia con 'V' e finisce con 'a':
	\begin{code}
	SELECT cognome, nome, citta FROM Studente 
	WHERE citta LIKE 'V%a';
	\end{code}
	
\noindent Per ignorare maiuscole/minuscole si utilizza ILIKE.

\item[-] SIMILAT TO -> è una versione più estesa del LIKE che accetta un sottoinsieme delle espressioni POSIX (versione SQL). Le stringhe di controllo compongono con:
	\begin{itemize}
	\item[-] "\_" == 1 carattere qualsiasi;
	\item[-] "\%" == 0 o più caratteri qualsiasi;
	\item[-] "\{n, m\}" == ripetizione del precedene match almeno n e max m volte;
	\item[-] "[...]" == elenco di caratteri ammissibili.
	\end{itemize}

	\noindent Esempio:
	
	\noindent Studenti con cognome che inizia con 'A', 'B',  'D' o 'N' e finisce con 'a':
	\begin{code}
	SELECT cognome, nome, citta FROM Studente
	WHERE cognome SIMILAR TO '[ABDN]{1}%a';
	\end{code}

\end{itemize}
\end{itemize}
\\\\
\textbf{ORDER BY} & ordina le tuple risultato in ordine rispetto agli attributi specificati. Si specifica come ultima clausola. La struttura è:
\begin{code}
	ORDER BY attributo [ASC | DESC], attributo2 [ASC | DESC], ...;
\end{code}
\noindent Le tuple vengono ordinate secondo il primo attributo, se vi sono valori uguali sul primo, si ordina basandosi sul secondo, e così via.
\\\\
\textbf{Operatori Aggregati} & permettono di determinare un valore considerando i valori ottenuti da una SELECT. Gli operatori aggregati possono ritornare un:
\begin{itemize}
\item[-] Conteggio:
\begin{code}
	 COUNT({ * | expr | ALL expr | DISTINCT expr }])
\end{code}
\noindent Il parametro expr è un'espressione che usa attributi e funzioni di attributi (non può usare operatori di aggregazione). Nello specifico:
	\begin{itemize}
	\item[-] COUNT(*) -> ritorna il numero di tuple che risultano dallla query;
	\item[-] COUNT(expr) -> ritorna il numero di tuple che rispettano expr (il valore di expr non è null);
	\item[-] COUNT(ALL expr) -> alias a COUNT(expr);
	\item[-] COUNT(DISTINCT expr) -> ritorna il numero di tuple che rispettano expr eliminando i doppioni.
	\end{itemize}
\noindent In genere COUNT ritorna una tabella composta da una sola colonna e una sola riga.
\item[-] Valore numerico/alfanumerico:
\begin{code}
	 SUM | AVG | MAX | MIN ({expr | DISTINCT expr }]) 
\end{code}
\end{itemize}

\end{longtable}
\noindent Il risultato della SELECT:
\begin{itemize}
\item[-] Ha come schema tutti gli attributi richiesti nella SELECT;
\item[-] Ha come contenuto tutte le tuple t ottenute proiettando sugli attributi
dopo SELECT le tuple $t_0$ appartenenti al prodotto cartesiano delle tabelle
ottenute dopo il FROM che soddisfano l’eventuale condizione nella
clausola WHERE / HAVING / GROUP BY.
\end{itemize}

\paragraph*{Funzioni usabili in expression} Il campo expression della SELECT permette anche di applicare delle funzioni alle colonne richieste. Alcune di queste sono:
\begin{itemize}
\item[-] UPPER(attributo) -> ritorna il contenuto tutto in maiuscolo;
\item[-] LOWER(attributo) -> ritorna il contenuto tutto in minuscolo;
\item[-] CHAR\_LENGHT(attributo) -> ritorna la lunghezza in caratteri dell'attributo;
\item[-] CURRENT\_DATE -> ritorna la data corrente in formato DATE;
\end{itemize}

\paragraph*{Conversioni nelle query} SQL permette di effettuare cast nei valori ritornati da una query. Il comando SQL per effettuarlo è:
\begin{code}
	CAST (expr AS newType)
\end{code}
\noindent PostgreSQl permette di abbreviarlo come segue:
\begin{code}
	expr::nuovoTipo
\end{code}

\noindent Esempio:

\noindent Calcola la media delle medie distinte degli studenti.

\begin{code}
	SELECT AVG(DISTINCT media)::DECIMAL(5,2) 
	FROM Studente;
\end{code}

\end{document}