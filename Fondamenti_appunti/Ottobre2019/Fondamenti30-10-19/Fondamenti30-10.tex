\documentclass[a4paper, 10pt]{report}
\usepackage[italian]{babel}
\usepackage[T1]{fontenc}
\usepackage[utf8]{inputenc}
\usepackage{charter}
\usepackage{amsmath}
\usepackage{amsthm}
\usepackage{amsfonts}
\usepackage{graphicx}
\usepackage{wrapfig}
\usepackage{tcolorbox}
\usepackage{fancyhdr}
\usepackage{tikz}
\usepackage{longtable}

\usepackage{geometry}
\geometry{a4paper, left=2cm,right=2cm,top=2cm,bottom=2cm}

\pagestyle{fancy}
\lhead{}
\chead{}
\lhead{\bfseries Fondamenti dell'informatica}
\rhead{\bfseries 28 ottobre 2019}

\begin{document}

\subsection*{Forme normali di una grammatica}
Esistono più modi per restringere la forma delle produzioni di una grammatica. I principali sono la forma normale di Chomsky e la forma normale di Greibach. Per arrivarci è necessaro introdurre alcune proprietà.

\subsubsection*{Eliminazione dei simboli inutili} Sia $G = <V, T, P, S>$ una grammatica context free. Un simbolo è utile se esiste una derivazione del tipo
\begin{align*}
S \rightarrow^* \alpha x \beta \rightarrow^* x \in T^*
\end{align*}
\noindent dove $\alpha, \beta \in (V \cup T)^*$. I simboli inutili possono essere eliminati in due passaggi (due lemmi).
Il primo passo consiste nell'eliminare tutte le variabili che non conducono in nessun modo ad una stringa di terminali:

\paragraph*{Lemma 1}: Sia $G = <V, T, P, S>$ una grammatica context free con $\mathfrak{L}(G) \ne 0$, allora esiste una grammatica equivalente $G' = <V', T, P', S>$ tale che per ogni $A \in V'$ esiste $x \in T^*$ dove
\begin{align*}
A \rightarrow^* x' \hspace{0.5cm} e \hspace{0.5cm} \mathfrak{L}(G) = \mathfrak{L}(G)
\end{align*}

\begin{tcolorbox}[title=\textbf{Dimostrazione}]
Dimostriamo per costruzione che il lemma funziona, ossia produciamo un algoritmo in grado di generare tale grammatica. 
	
	Definiamo la funzione $\Gamma:\mathcal P(V)\to\mathcal P(V)$ nel seguente modo: 
	\[
		\Gamma(W) = \{A\in V:\exists\alpha\in(T\cup W)^*.A\rightarrow^*\alpha\in P\}
	\]
	La funzione prende in input un insieme di simboli non terminale e restituisce in output un insieme di simboli non terminali dai quali è possibile raggiungere un simbolo terminale o un simbolo presente nell'insieme di input. 
	
	Applicando la funzione all'insieme vuoto avremo nell'insieme i soli simboli che direttamente portano ad un simbolo terminale, applicando nuovamente la funzione a questo insieme aggiungeremmo altri simboli, e così via, fino a che non arriveremmo al punto fisso, in cui avremmo aggiunto tutti i simboli ed ogni successiva applicazione non ne aggiungerà di nuovi. Questo può essere verificato per induzione. 
	
	Definiamo quindi per casi la funzione $\Gamma^{(n)}(W)$ che ci consentirà di applicare $\Gamma$ $n$ volte:
	\[
		\begin{cases}
			\Gamma^0(W)&= W \\
			\Gamma^{n+1}(W)&=\Gamma(\Gamma^{n+1}(W))
		\end{cases}
	\]
	
	Notiamo che nel caso pessimo ogni iterazione andrà ad aggiungere un solo simbolo all'insieme per cui dovremmo applicare $\Gamma$ al più un numero di volte pari alla cardinalità dell'insieme dei simboli non terminali per essere certi di aver incluso tutti i simboli.
	
	Per cui, possiamo definire i nuovi insiemi $V'$ e $P'$ nel seguente modo:
	\[
		\begin{aligned}
			V'&=\Gamma^{|V|}(\emptyset)\\
			P'&=\{A\to\alpha\in P: A\in V' \}
		\end{aligned}
	\]
\end{tcolorbox}

\noindent Il secondo passo consiste nell'eliminare tutte le variabili che non sono mai raggiunte da $S$:


\paragraph*{Lemma 2}: Sia $G = <V, T, P, S>$ una grammatica context free, allora esiste una grammatica equivalente $G' = <V', T', P', S'>$ tale che per ogni $x \in (V' \cup T')$ esistono $\alpha$ e $\beta$ in $(V' \cup T')$ dove
\begin{align*}
S \rightarrow^* \alpha x \beta \hspace{0.5cm} e \hspace{0.5cm} \mathfrak{L}(G) = \mathfrak{L}(G)
\end{align*}

\begin{tcolorbox}[title=\textbf{Dimostrazione}]
$\Gamma: P(V \cup T) \rightarrow P(V \cup T)$

$\Gamma(w) = \{x \in V \cup T \cup \{\epsilon\} : \exists A \in w . A \rightarrow \alpha x \beta \in P\} \cup {S}$\\

La soluzione per la dimostrazione è la raggiungibilità da parte di S, miriamo a ottenere tutti i simboli raggiungibili da S (che coincidono con l'insieme w).\\

$\Gamma(\emptyset) = {S}, \Gamma^2(\emptyset) = {A, a, B, S}, \Gamma(\emptyset)^3 = {A, a, B, S, C}$\\

Si può aggiungere un simbolo alla volta, nel caso peggiore si termina in un numero di passi uguale al numero di simboli.
\end{tcolorbox}

\paragraph*{Teorema} Ogni linguaggio context free non vuoto è generato da una grammatica context free priva di simboli inutili.

\begin{tcolorbox}[title=\textbf{Dimostrazione}]
Applichiamo alla grammatica G in sequenza il lemma 1 e il lemma 2. 
	\[
		G\xrightarrow{lemma 1}G_1\xrightarrow{lemma 2}G_2
	\]\end{tcolorbox}

\end{document}