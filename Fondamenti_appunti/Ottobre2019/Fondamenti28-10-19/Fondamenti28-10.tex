\documentclass[a4paper, 10pt]{report}
\usepackage[italian]{babel}
\usepackage[T1]{fontenc}
\usepackage[utf8]{inputenc}
\usepackage{charter}
\usepackage{amsmath}
\usepackage{amsthm}
\usepackage{amsfonts}
\usepackage{graphicx}
\usepackage{wrapfig}
\usepackage{tcolorbox}
\usepackage{fancyhdr}
\usepackage{tikz}
\usepackage{longtable}

\usepackage{geometry}
\geometry{a4paper, left=2cm,right=2cm,top=2cm,bottom=2cm}

\pagestyle{fancy}
\lhead{}
\chead{}
\lhead{\bfseries Fondamenti dell'informatica}
\rhead{\bfseries 28 ottobre 2019}

\begin{document}

\section*{\underline{Grammatiche libere dal contesto}}
Una grammatica può essere intesa come un insieme di regole che permettono di generare un linguaggio. Le grammatiche libere dal contesto permettono di descrivere la sintassi dei linguaggi di programmazione.\\

\noindent Una \textbf{grammatica} può essere definita come una quadrupla $<V, T, P, S>$ (per la spiegazione vedere linguaggi).\\

\noindent Una \textbf{produzione di un linguaggio context free} è del tipo:
\begin{align*}
A \rightarrow \alpha
\end{align*}
\noindent ovvero in ogni stringa posso sostituire $A$ con $\alpha$ indipendentemente da ciò che ho a destra e a sinistra\footnote{Da questo deriva il nome "libere dal contesto"}.\\

\noindent Esempio:

\begin{align*}
\gamma A \delta A w \rightarrow^1 \gamma \alpha \delta A \rightarrow^2 \gamma \alpha \delta \alpha w
\end{align*}

\subsection*{Linguaggio libero dal contesto}
Un linguaggio generato da una grammatica $G$ si definisce come segue:
\begin{align*}
L(G) = \{x \in T^* | S \rightarrow^* x\}
\end{align*}
\noindent ovvero consiste di tutte quelle stringhe che, a partire dal simbolo iniziale e dopo un numero finito di riduzioni ($\rightarrow^*$), arrivano ad $x$.\\

\noindent Un linguaggio si dice libero dal contesto se è generato da una grammatica libera dal contesto. Un linguaggio può essere generato da infinite grammatiche.

\paragraph*{Definizione} Una grammatica si dice ambigua se esiste una stringa per la quale esistono due alberi di derivazione diversi.\\

\noindent \underline{Esempio: dimostrare che il linguaggio è context free}
\begin{align*}
\mathfrak{L} = \{ a^n b^n | n \ge 0\}
\end{align*}

\noindent Per dimostrare che il linguaggio è CE devo costruire la grammatica CE che lo genera.
\begin{align*}
T =& \{ a, b \}\\
V =& \{ S\}\\
S =& iniziale
\end{align*}

\noindent Ora devo costruire le produzioni:
\begin{itemize}
\item[-] Poichè posso avere anche $n = 0$, il linguaggio contiene la string avuote ($\varepsilon = a^0b^0$);
\item[-] Se $ a^nb^n$ ($S \rightarrow a^nb^n$) è una stringa del linguaggio, allora anche $aSb$ appartiene al linguaggio ($== aa^nb^nb$).
\end{itemize}

\noindent Quindi, schematizzando:
\begin{align*}
P =
\begin{cases} 
S \rightarrow \varepsilon \\ 
S \rightarrow aSb
\end{cases} 
\hspace{1cm} oppure \hspace{1cm} P = \varepsilon | aSb
\end{align*}

\noindent Ora devo dimostrare che il linguaggio generato dalla grammatica che ho costruito è uguale a quello di partenza ($\mathfrak{L}(G) = \mathfrak{L}$):\\

\noindent \textbf{Dimostrazione $\mathfrak{L}(G) \leftarrow \mathfrak{L}$}

\begin{longtable}{p{.40\textwidth} p{.40\textwidth}}
Caso $n = 0 \rightarrow \varepsilon \in \mathfrak{L}$ & $S \rightarrow^1 \varepsilon \in \mathfrak{L}(G)$
\\\\
Caso $n > 0 \rightarrow a^nb^n \in \mathfrak{L}$ e $a^nb^n \in \mathfrak{L}(G)$ suppongo $S \rightarrow^* a^nb^n$ & $S \rightarrow aSb \rightarrow^* aa^nb^nb = a^{n+1}b^{n+1}$
\end{longtable}

\noindent \textbf{Dimostrazione $\mathfrak{L}(G) \rightarrow \mathfrak{L}$}
\begin{longtable}{p{.40\textwidth} p{.40\textwidth}}
Caso $n = 1 $ & $ \varepsilon \in \mathfrak{L}$
\\\\
Caso $n > 1 $ suppongo $ (S \rightarrow^n a^kb^k)$ & $\underline{S \rightarrow aSb \rightarrow^n}_{n+1} \hspace{0.2cm} aa^kb^kb$
\end{longtable}

\noindent \underline{Esempio: costruire la grammatica}
\begin{align*}
\mathfrak{L} = \{a^mb^{n+2}c^na^{m+2} | m, n \ge 0\}
\end{align*}

\noindent Scompongo la stringa principale in due sottostringhe $b^{n+2}c^n$ e $a^ma^{m+2}$:\\

$S \rightarrow aAaaa | Aaa$ (caso stringa vuota)

$A \rightarrow bbbAc | bb$ (caso stringa vuota)\\

\noindent Questa soluzione presenta però un errore: ?. Quindi Riscrivo la stringa principale come $a^m[bb(b^nc^n)aa]a^n$:\\

$S \rightarrow aSa | bbBaa$

$B \rightarrow \varepsilon | bBc$ ($B == b^nc^n$)
\end{document}